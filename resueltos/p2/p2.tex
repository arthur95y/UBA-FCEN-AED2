\documentclass[10pt,a4paper]{article}

\input{AEDmacros}
\usepackage{caratula} % Version modificada para usar las macros de algo1 de ~> https://github.com/bcardiff/dc-tex


\titulo{Resolucion de la Practica 2 en \LaTeX}
\subtitulo{@AY022}



\materia{Algoritmos y Estructuras de Datos}

% Pongan cuantos integrantes quieran

% Declaramos donde van a estar las figuras
% No es obligatorio, pero suele ser comodo
\graphicspath{{../static/}}

\begin{document}

\maketitle
\section{Funsiones auxiliares}
\subsection{Ejercicio1}

Escriba los siguientes predicados sobre números enteros en lenguaje de especificación:

\vspace{0.3cm}

a) pred esCuadrado(x:$\ent$) que sea verdadero si y solo si x es un numero cuadrado 

\textbf{pred esCuadrado} (x:$\ent$) \{

($\exists$y:$\ent$)(y^2 = x $\wedge$ x $\geq$ 0) \}


\vspace{0.3cm}

b) pred esPrimo (x:\ent) que sea verdadero sii x es primo 

\textbf{pred esPrimo} (x:\ent) \{

 x $\geq$ 2 $\wedge$ ($\forall$y:$\ent$)(2 $<$ y $<$ x $\implies$ mod x y $\neq$ 0)\}

 \vspace{0.3cm}

 c) pred sonCoprimos (x,y :$\ent$) que sea verdadero sii son coprimos.

 \textbf{pred sonCoprimos} (x,y:$\ent$) \{

$\neg$($\exists$ z: $\ent$)(2 $\leq$ z $\wedge$ mod x z == 0 $\wedge$ mod y z == 0)\}

\vspace{0.3cm}

d) pred mayorPrimoQueDivide (x,y:$\ent$) que sea verdadero sii el mayor primo divide a x 

\textbf{pred mayorPrimoQueDivide} (x,y:$\ent$) \{ 

esPrimo(y) $\wedge$ mod x y == 0 $\wedge$ $\neg$($\exists$ z: $\ent$)(esPrimo(z) $\wedge$ z $>$ y $\wedge$ mod x z == 0)   \} 

\subsection{Ejercicio2}

Escriba los siguientes predicados auxiliares sobre secuencias de enteros, aclarando los tipos de los parámetros
que recibe

\vspace{0.3cm}

a)esPrefijo, que determina si una secuencia es prefijo de otra. 

\textbf{pred esPrefijo} (pre: seq$<$\ent$>$, s: seq$<$\ent$>$) \{

$|$pre$|$ $<$ $|$s$|$ $\wedge$ subseq(s,0,$|$pre$|) == pre \}

\vspace{0.3cm}

b)estaOrdenada, que determina si la secuencia esta ordenada de menor a mayor.

\textbf{pred estaOrdenada}(s: seq$<$\ent$>$) \{

($\forall$i:\ent)(0 $\leq$ i $<$ $|$s$|$-1 $\implies$ $s_i$  $<$  $s_{i+1}$ ) \}

\vspace{0.3cm}

c)hayUnoParQueDivideAlResto, que determina si hay un elemento par en la secuencia que divide a todos los otros 

elementos de la secuencia

\textbf{pred hayUnoParQueDivideAlResto} (s: seq$<$\ent$>$) \{

($\exists$i: $\ent$) (0 $\leq$ i $<$ $|$s$|$ $\wedge$ mod $s_i$ 2 == 0 $\wedge$ ($\forall$j: $\ent$)(0 $\leq$ j $<$ $|$ s $|$ $\implies$ mod $s_j$ $s_i$ == 0))

\vspace{0.1cm}
\textbf{¡IMPORTANTE!}: esto nos dice literalmente \textbf{\underline{"hay al menos un elemento que es par}} y que divide a 

¡¡todos los demas!!". (donde lo que esta en negrita es debido al primer cuantificador existencial que es el mas importante 

de los dos cuantificadores en mi opinion).

El concepto anterior y el opuesto que seria  \underline{\textbf{"para todos los elementos}} implica que ¡¡existe uno!! que bla bla.." 


 (donde ahora el primer cuantificador es el para todo y es el mas importante) son buenisimos.

\vspace{0.3cm}

d)sinRepetidos, que determina si la secuencia no tiene repetidos

\textbf{pred sinRepetidos}(s: seq$<$\ent$>$) \{

($\forall$ i,j: $\ent$)( 0 $\leq$ i,j $<$ $|$s$|$ $\wedge$ i $\neq$ j $\implies$ $s_i$ $\neq$ $s_j$ ) \}

\vspace{0.3cm}

e) enTresPartes,que determina si en la secuencia aparecen (de izquierda a derecha) primero 0s, después 1s y por último 

2s. Por ejemplo ⟨0, 0, 1, 1, 1, 1, 2⟩ cumple con enTresPartes, pero ⟨0, 1, 3, 0⟩ o ⟨0, 0, 0, 1, 1⟩ no. ¿Cómo modificarı́a la 

expresión para que se admitan cero apariciones de 0s, 1s y 2s (es decir, para que por ejemplo ⟨0, 0, 0, 1, 1⟩ o ⟨⟩ sı́ 

cumplan enTresPartes)?

\textbf{pred enTrePartes} (s: seq $<$$\ent$$>$) \{ 

($\exists$ i,j,k:$\ent$)(0 $\leq$ i,j,k $<$ $|$s$|$ $\wedge$ i$\leq$ j $\leq$k  $\wedge$ $|$k$|$=$|$s$|$ $\wedge$ sonTodos(0,subseq(s,0,i)) $\wedge$ sonTodos(1,subseq(s,i,j)) $\wedge$ sonTodos(2,subseq(s,j,k)))\}

\vspace{0.1cm}

\textbf{¡IMPORTANTE!}: el ejercicio 2(a) y este, que usan subsecuencias, ayudan muchisimo.


\vspace{0.5cm}

\textbf{Estos conceptos resaltamos son muy importantes saberlos y tenerlos en cuenta porque para especificacion,}

\textbf{TADs (se vera mas adelante) y para el segundo parcial (invariante de representacion y funcion de .}

\textbf{abstraccion) ayudara muchisimo (al menos para mi).}




\end{document}



%% modulo          $|$ $|$
%% subIndices      $s_{i+1}$
%% {   }           \{ \}
%% >               $>$
%% <               $<$
%% menor igual     \leq
%% mayor igua      \geq
%% distinto        \neq
%% seq:(Entero)    \TLista{\ent}
%% al cuadrado ^2        
%% implica     \implies
%% "subrayado" \underline
%% "negrita"   \textbf{}
%% "espacios"  \vspace
%% o           \vee
%% y           \wedge
%% para todo   \forall
%% existes     \ exists 
%% Entero      \ent

	\begin{lstlisting}[caption={Ejemplo de código (usando los estilos de la cátedra, ver las macros para más detalles)},label=code:for]
res := 0;
i := 0;
while (i < s.size()) do
	res := res + s[i];
	i := i + 1
endwhile
	\end{lstlisting}
%\end{minipage}

