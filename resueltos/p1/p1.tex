\documentclass[10pt,a4paper]{article}

\input{AEDmacros}
\usepackage{caratula} % Version modificada para usar las macros de algo1 de ~> https://github.com/bcardiff/dc-tex


\titulo{Resueltos de la Practica 1 en \LaTeX}
\subtitulo{}

\fecha{\today}

\materia{Algoritmos y Estructura de Datos II}


\integrante{Yale Quispe Arthur Jr}{1573/21}{yalequispe@gmail.com}


% Pongan cuantos integrantes quieran

% Declaramos donde van a estar las figuras
% No es obligatorio, pero suele ser comodo
\graphicspath{{../static/}}

\begin{document}

\maketitle

\section{Repaso de logica proposicional}
\subsection{Ejercicio1}

Determinar los valores de verdad de las siguientes proposiciones cuando el valor de verdad de \textbf{a}, \textbf{b} y \textbf{c} es verdadero y el de \textbf{x} e \textbf{y} es falso.

\vspace{0.3cm}

a) (\neg x \vee b) =

(\neg false \vee true ) = (true \vee (no importa el valor de verdad)) ) = true 

\vspace{0.3cm}

b) ((c \vee (y \wedge a)) \vee b) =

((true \vee  (no importa el valor de verdad) \vee (no importa el valor de verdad))) = true  

\vspace{0.3cm}

c) \neg(c \vee y) =

 \neg(true \vee false) = \neg true = false

\vspace{0.3cm}

d) \neg (y \vee c) = false

sol : false, pues vale la conmutatividad del inciso anterior

\vspace{0.3cm}

e) (\neg(c \vee y) \iff (\neg c \wedge \neg y)) = true 

sol : son equivalentes, pues si hacemos la distributividad de la negacion, en un termino, te queda igual al otro termino 

\vspace{0.3cm}

f) ((c \vee y) \wedge (a \vee b)) =

(true \wedge true) = true 

\vspace{0.3cm}

g) ((c \vee y) \wedge (a \vee b)) \iff (c \vee (y \wedge a) \vee b ) =

 true \iff true = true

\vspace{0.3cm}

h) (\neg c \wedge \neg y) = false 

sol : pues es resuelto en el inciso e

\subsection{Ejercicio2}

Considere la siguiente oración: “Si es mi cumpleaños o hay torta, entonces hay torta”.
\begin{itemize}
	\item a) Escribir usando lógica proposicional y realizar la tabla de verdad
	\item b) Asumiendo que la oración es verdadera y hay una torta, qué se puede concluir?
	\item c) Asumiendo que la oración es verdadera y no hay una torta, qué se puede concluir?
        \item d) Suponiendo que la oración es mentira (es falsa), se puede concluir algo?
\end{itemize}

\vspace{0.3cm}

solucion:

\vspace{0.3cm}

a)c = cumple y t = torta 

(c \lor t) \implies t 

\vspace{0.3cm}

b) Haciendo la tabla de verdad, se puede ver que hay dos casos que cumple esta situacion:

1- hay cumple 

2- no hay cumple

\vspace{0.3cm}

c) Viendo la tabla de verdad, se puede conclur que: solo ocurre cuando "no hay cumple".

\vspace{0.3cm}

d) Viendo la tabla, se concluye que: hay cumple  $\wedge$    no hay torta.

\subsection{Ejercicio5}

\vspace{0.3cm}

\underline{Mi observacion sobre el tema Relacion de Fuerza:} 

\underline{Este concepto lo exploraremos en el futuro, especificamente en el capitulo 3, "La Precondicion mas Debil". Esta} 

\underline{WP \textbf{es la precondicion menos exigente de todas y nos dira qué necesita como minimo una precondicion 'P'}  }

\underline{con respecto a un programa \textbf{'S'} y su postcondicion/asegura \textbf{'Q'}, ya dados, para que sea valida o correcta la tripla.}

\vspace{0.3cm}

\underline{Esto quiere decir: Si nos dan un Programa \textbf{'S'}, una Postcondicion \textbf{'Q'} y hayamos (o tenemos) la WP (weakest precondition)}

\underline{¡¡ esta precondicion será la mejor de todas!!. Entonces, a partir de esta WP podemos saber si otras precondiciones son o no }

\underline{correctas (ya que pueden haber muchas precondiciones, pero WP es unica, en mi entender), siempre y cuando las  }

\underline{ precondiciones \textbf{impliquen} la WP o dicho de otra forma  (\textbf{'P'} $\implies$ WP(\textbf{'S'},\textbf{'Q'}) ) sea una tautologia }

\vspace{0.3cm}

\underline{notar que: como la WP es la proposicion mas fuerte, cualquier precondicion que haga verdadera la implicacion es }

\underline{suficiente para afirmar que es correcta la tripla.}

\vspace{0.3cm}

Enunciado: 

Asumiendo que el valor de verdad de b y c es verdadero, el de a es falso y el de x e y es indefinido, indicar

cuáles de los operadores deben ser operadores “luego” para que la expresión no se indefina nunca:

\vspace{0.3cm}

a) True, False

T $\implies$ F = F 

No hay tautologia, por lo tanto $"False"$ es mas fuerte que $"True"$ (siempre "False" es mas fuerte, pues si el

antecedente es "False" no importa lo que haya en el consecuente siempre sera tautologia)

\vspace{0.3cm}

b) (p $\wedge$ q), (p $\vee$ q)


(p $\wedge$ q) \implies (p $\vee$ q) = Tautologia? 

Si, se puede ver haciendo la tabla de verdad.

Por lo tanto (p $\wedge$ q) es mas fuerte que (p $\vee$ q).

\vspace{0.3cm}

c) p, (p $\wedge$ q)

p $\implies$ (p $\wedge$ q) = Tautologia ?

No (por tabla), pero ( (p $\wedge$ q) $\implies$ p = Tautologia )

por lo tanto (p $\wedge$ q) es mas fuerte 

\vspace{0.3cm}

d) p, (p $\vee$ q)

( p $\implies$ (p $\vee$ q) ) = Tautologia ? 

si (por tabla), entonces p es mas fuerte.



\vspace{0.3cm}

e) p , q 

(p $\implies$ q ) = Tautologia?

No (por tabla), en ninguno de los casos.



\vspace{0.3cm}

f) p, (p $\implies$ q)

( p $\implies$ (p $\implies$ q) ) = Tautologia ? 

No (por tabla), en ninguno de los casos.

\vspace{0.3cm}

\textbf{En esta guia priorice los ejercicios estrellas, de especificacion y relacion de fuerza}

\textbf{No me centre en reglas de equivalencias o en operadores "luego" }






% IMPORTANTE (esta parte de aca) : 
% \vee "o" logico

% \wedge  "y" logico
%-----------------------
% \leq menor igual

% \geq mayor igual
%-----------------------
% \neg  negacion



\end{document}
